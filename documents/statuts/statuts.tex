\documentclass[a4paper,defaultorg]{organisation-contract}
\title{Statuts de l'association}

\atlocation{à Grenoble}


\begin{document}
\section{Constitution}
L'association dite : \og \orgshort{} \fg{}
fondée en \orgcreatedin{}
a pour but <INSERT HERE>

Sa durée est illimitée.

Elle a son siège social au \orgsinglelineaddress.

\section{Moyens d'action}
Les moyens d'action de l'association sont <INSERT HERE>

\section{Composition}
L'association se compose de membres adhérents, ... 
Les membres adhérents peuvent être des personnes morales légalement consituées,
conformément à l'article 5 de la loi du 1er juillet 1901.

Pour être membre, il faut être agrée par le conseil d'administration.

La cotisation annuelle est <INSERT HERE>


Le titre de membre honoraire peut être décerné par le conseil d'administration
aux personnes qui rendent ou qui ont rendu des services signalés à
l'association. Ce titre confère aux personnes qui l'ont obtenu le droit de faire
partie de l'assemblée générale sans être tenues de payer une cotisation.

\section{Perte de la qualité de membre}
La qualité de membre de l'association se perd:

\begin{itemize}
    \item par la démission;
    \item par la radiation prononcée pour non-paiement de la cotisation ou pour
        motifs graces par le conseil d'administration, sauf recours à
        l'assemblée générale. Le membres intéressé est préalablement appelé à
        forunir ses explications.
\end{itemize}

\section{Administration}
L'association est administrée par un conseil d'administration dont le nombre de
membres, fixé par délibération de l'assemblée générale, est compris entre 2
membres au moins et 11 membres au plus. Les membres du conseil sont élus au
scrutin secret, pour 2 ans, par l'assemblée générale et choisis dans les
catégories de membres dont se compose cette assemblée.

En cas de vacance, le conseil pourvoit provisoirement au remplacement de ses
membres. Il est procédé à leur remplacement définitif par la plus prochaine
assemblée générale.

Les pouvoirs des membres ainsi élus prennent fin à l'époque où devrait
normalement expirer le mandat des membres remplacés.

Le renouvellement du conseil à lieu par moitié.

Les membres sortants sont rééligibles.

Chaque administateur ne peut détenir plus d'un pouvoir.

Le conseil choisit parmi ses membres, au scrutin secret, un bureau composé d'un
président, d'un vice-président, d'un secrétaire et d'un trésorier. 
% les effectifs du bureau ne doivent pas excéder le tiers de ceux du conseil

Le bureau est élu pour 1 an.

\section{Réunion du conseil d'administration}
Le conseil se réunit une fois au moins tous les six mois et chaque fois qu'il
est convoqué par son président ou sur la demande du quart des membres de
l'association.

La précence du tiers au moins des membres du conseil est nécessaire pour la
validité des délibérations.

En cas de partage des voix, celle du président est prépondérante.

Il est tenu procès-verbal des séances.

Les procès-verbaux sont signés par le président et le secrétaire. Ils sont
établis sans blancs, ni ratures, sur des feuillets numérotés et conservés au
siège de l'association.

\section{Rétribution}
Les membres du conseil d'administration ne peuvent recevoir aucune rétribution
à raison des fonctions qui leurs sont confiées.

Des rembousements de frais sont seuls possibles. Ils doivent faire l'objet d'une
décision expresse du conseil d'administration, statuant hors de la présence des
intéressés; des justifications doivent être produites qui font l'objet de
vérifications.

Les agents rétribués de l'association peuvent être appelés par le président à
assister avec voix consultative, aux séances de l'assemblée générale et du
conseil d'administration.

\section{Assemblée générale}

L'assemblée génerale de l'association comprend les membres adhérents, les
membres honoraires. Dans le cas où le membre adhérent est une personne morale
régulièrement constituée, il est accordé une voix délibérative au représentant
légal ou délegué.
L'assemblée générale se réunit au moins une fois par an et chaque fois qu'elle
est convoquée par le conseil d'administration ou sur la demande du quart au
moins des membres de l'association.
Son ordre du jour est réglé par le conseil d'administration. Elle choisit son
bureau qui peut être celui du conseil d'administration. Elle entend les rapports
sur la gestion du conseil d'administration, sur la situation financière et
morale de l'association. Elle approuve les comptes de l'exercice clos, vote le
budget de l'exercice suivant, délibère sur les questions mises à l'ordre du jour
et pourvoit, s'il y a lieu au renouvellement des membres du conseil
d'administration.

Il est tenu procès-verbal des séances. Les procès-verbaux sont signés par le
président et le secrétaire. Ils sont établis sans blancs, ni ratures, sur des
feuillets numérortés et conservés au siège de l'association.

Chaque membre présent ne peut détenir plus de 2 pouvoirs en sus du sien. En cas
de partage des voix, celle du président est prépondérante.

Le rapport annuel des comptes sont adressés chaque année à tous les membres de
l'association.

Sauf application des dispositions de l'article précédent, les agents rétribués,
non membres de l'association n'ont pas accès à l'assemblée générale.

\section{Représentation}
Le président représente  l'association dans tous les actes de la vie civile. Il
ordonnance les dépenses. Il peut donner délégation dans les conditions qui sont
fixées par le règlement intérieur.
En cas de représentation en justice, le président ne peut être remplacé que par
un mandataire agissant en vertu d'une procuration spéciale.
Les représentants de l'association doivent jouir du plein exercice de leurs
droits civils.

\section{Immobilier}
Les délibérations du conseil d'administration relatives aux acquisitions,
échanges et alienations d'immeubles nécessaires au but poursuivi par
l'association, constitutions d'hypothèques sur lesdits immeubles, baux exédant
neuf années, alienations de biens rentrant dans la dotation et emprunts doivent
être approuvées par l'assemblée générale.

\section{Dons et legs}
L'acceptation des dons et legs par délibération du conseil d'administration
prend effet dans les conditions prévues par l'article 910 du code civil. Les
délibération de l'assemblée générale relatives aux aliénaations de biens
mobiliers et immobiliers dépendant de la dotation, à la constitutions
d'hypothèques et aux emprunts, ne sont valables qu'après approbation
administrative.

\section{Fonctionnement}
%% pouvoirs conférés aux personnes chargées de la direction

\section{Dotation}
La dotation comprend :
\begin{enumerate}
    \item une somme de < > (mentionner ici les capitaux mobiliers faisant partie de la dotation au moment de la
        demande) constituée en valeurs placées conformément aux prescriptions de l'article suivant ;
    \item les immeubles nécessaires au but recherché par l'association ainsi que des bois, forêts ou terrains à
        boiser ;
    \item les capitaux provenant des libéralités, à moins que l'emploi immédiat n'en ait été décidé ;
    \item les sommes versées pour le rachat des cotisations ;
    \item le dixième au moins, annuellement capitalisé, du revenu net des biens de l'association ;
    \item la partie des excédents de ressources qui n'est pas nécessaire au fonctionnement de l'association pour
        l'exercice suivant.
\end{enumerate}

\section{}
Tous les capitaux mobiliers, y compris ceux de la dotation, sont placés en
titres nominatifs, en titres pour lesquels est établi le bordereau de références
nominatives prévu à l'article 55 de la loi numéro 87-416 du 17 juin 1987 sur
l'épargne ou en valeurs admises par la Banque de France en garantie d'avance.

\section{}
Les recettes annuelles de l'association se composent :
\begin{enumerate}
    \item du revenu de ses biens à l'exception de la fraction prévue au 5 de
        l'article 13 ;
    \item des cotisations et souscriptions de ses membres ;
    \item des subventions de l'Etat, des régions, des départements, des
        communes et des établissements publics ;
    \item du produit des libéralités dont l’emploi est décidé au cours de l’exercice ;
    \item des ressources créées à titre exceptionnel et, s'il y a lieu, avec
        l'agrément de l'autorité compétente
    \item du produit des ventes et des rétributions perçues pour service rendu.
\end{enumerate}

\section{Comptabilité}
Il est tenu une comptabilité faisant apparaître annuellement un compte de
résultat, un bilan et une annexe.  Chaque établissement de l'association doit
tenir une comptabilité distincte qui forme un chapitre spécial de la
comptabilité d'ensemble de l'association (lorsque l'association possède ou se
propose de créer des comités locaux, cette règle doit être étendue par une
disposition des statuts).  Il est justifié chaque année auprès du préfet du
département, du ministre de l'intérieur et du ministre < > (indiquer le(s)
ministre(s) au département duquel (desquels) ressortit l'association) de
l'emploi des fonds provenant de toutes les subventions accordées au cours de
l'exercice écoulé.

\section{Modification des statuts}
\label{modification}
Les statuts peuvent être modifiés par l'assemblée générale sur la proposition
du conseil d'administration ou sur la proposition du dixième des membres dont
se compose l'assemblée générale.  Dans l'un et l'autre cas, les propositions de
modifications sont inscrites à l'ordre du jour de la prochaine assemblée
générale, lequel doit être envoyé à tous les membres de l'assemblée au moins <
> jours à l'avance.  L'assemblée doit se composer du quart au moins des membres
en exercice. Si cette proportion n'est pas atteinte, l'assemblée est convoquée
de nouveau, mais à quinze jours au moins d'intervalle, et cette fois, elle peut
valablement délibérer, quel que soit le nombre des membres présents ou
représentés.  Dans tous les cas, les statuts ne peuvent être modifiés qu'à la
majorité des deux tiers des membres présents ou représentés.

\section{Dissolution de l'association}
\label{dissolution}
L'assemblée générale, appelée à se prononcer sur la dissolution de
l'association et convoquée spécialement à cet effet, dans les conditions
prévues à l'article précédent, doit comprendre, au moins, la moitié plus un des
membres en exercice.  Si cette proportion n'est pas atteinte, l'assemblée est
convoquée de nouveau, mais à quinze jours au moins d'intervalle, et cette fois,
elle peut valablement délibérer, quel que soit le nombre des membres présents
ou représentés.  Dans tous les cas, la dissolution ne peut être votée qu'à la
majorité de deux tiers des membres présents ou représentés.

\section{Liquidation de l'association}
\label{liquidation}
En cas de dissolution, l'assemblée générale désigne un ou plusieurs
commissaires, chargés de la liquidation des biens de l'association. Elle
attribue l'actif net à un ou plusieurs établissements analogues, publics, ou
reconnus d'utilité publique, ou à des établissements visés à l'article 6,
alinéa 5, de la loi du 1er juillet 1901 modifiée.6

\section{Approbation}
Les délibérations de l'assemblée générale prévues aux articles
\ref{modification}, \ref{dissolution} et \ref{liquidation} sont adressés, sans
délai, au ministre de l'intérieur et au ministre chargé de < >.  Elles ne sont
valables qu'après approbation du Gouvernement.

\section{Surveillance}
Le président doit faire connaître dans les trois mois, à la préfecture du
département ou à la sous-préfecture de l'arrondissement où l'association a son
siège social, tous les changements survenus dans l'administration ou la
direction de l'association.  Les registres de l'association et ses pièces de
comptabilité sont présentés sans déplacement, sur toute réquisition du ministre
de l'intérieur ou du préfet, à eux-mêmes ou à leur délégué ou à tout
fonctionnaire accrédité par eux.  Le rapport annuel et les comptes - y compris
ceux des comités locaux - sont adressés chaque année au préfet du département,
au ministre de l'intérieur et au ministre chargé de < >.


\section{}
Le ministre de l'intérieur et le ministre chargé de < > ont le droit de faire
visiter par leurs délégués les établissements fondés par l'association et de se
faire rendre compte de leur fonctionnement.

\section{Règlement intérieur}
Le règlement intérieur préparé par le conseil d'administration et adopté par
l'assemblée générale est adressé à la préfecture du département. 
%Il ne peut entrer en vigueur ni être modifié qu'après approbation du ministre de
%l'intérieur.

\vspace{4cm}

Fait à , le xx/xx/xxxx

    Rory Williams\hfill
    Amelia Pond\hfill
    River Song\hfill
    Harriet Jones

\end{document}
